
\chapter{Grundlagen und Stand der Technik}

\section{Motivation}
Gegenstand dieses Kapitels ist es, dem Leser einen Überblick über für die Arbeit relevante Grundlagen und verwandten Arbeiten zu geben. Folgende grundlegende Fragen sollten beantwortet werden:
\begin{itemize}
	\item Welche artverwandten Projektarbeiten oder Produkte existieren?
	\item Wo ist der Mangel zum Stand der Technik? Warum ist es notwendig oder relevant, das Thema in der Projektarbeit zu behandeln? 
	\item Wie grenzt sich der grundlegende Lösungsansatz vom Stand der Technik ab?
	\item Existieren Vorarbeiten, auf denen aufgebaut wird?
\end{itemize}

Die Motivation zur Projektarbeit kann vielfältig sein und sollte daher ausreichend begründet werden. Beispiele: Ein bestehendes Problem wurde durch den Stand der Technik bisher nicht oder nicht zufriedenstellend gelöst. Oder: Gegenstand der Arbeit ist eine besonders kostengünstige Lösung. Oder auch: Die Arbeit behandelt eine grundlegende Evaluation, um Möglichkeiten als auch Limitationen einer neuen Technologie aufzuzeigen.

\section{Stand der Technik}

Zum Stand der Technik oder Stand des Wissens gehören alle 

Als Startpunkt für einen ersten Überblick bietet sich die Internet-Suchmaschine Ihres Vertrauens an. Die Qualität der Ergebnisse hängt hierbei grundlegend von den verwendeten Schlüsselwörtern ab. 

Der erste Überblick hilft, weitere relevante Schlüsselwörter zu definieren. Mit diesen bietet sich eine gezieltere Suche in fachspezifischen Datenbanken an. Beispiele:

\begin{itemize}
\item IEEE Explore
\item CiteSeer http://citeseer.ist.psu.edu
\item arXiv http://arxiv.org
\item Google Schoolar
\item ...
\end{itemize}

Innerhalb der Hochschule verfügen Sie über einen kostenfreien Zugang zu Publikationen der IEEE - nutzen Sie diese qualitativ hochwertige Quelle! 

Bitte verwenden Sie zur Darlegung des Stands der Technik vorrangig {\emph nicht flüchtige Quellen}, beispielsweise Publikationen in Form von Buch-, Zeitungs-, oder Konferenzbeiträgen.
Vermeiden Sie nach Möglichkeit {\emph flüchtige Quellen}, beispielsweise Informationen, die ausschließlich durch Internetseiten, Blog-Einträge und dergleichen belegt sind. Der Leser muss die Möglichkeit haben, Ihre durch Literaturangaben hinterfütterten Aussagen noch nach Jahren nachvollziehen zu können - dies ist durch reiche Verwendung sekündlich veränderbarer Inhalte einzelner Internetseiten nicht gegeben.

