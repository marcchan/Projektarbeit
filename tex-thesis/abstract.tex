%
\chapter*{Kurzfassung}
%
\label{sec:Abstract}%
% add this page manually to the table of contents with ROMAN letters
\addcontentsline{toc}{section}{Abstract}
%
The aim of this thesis is to explore strategies for
real-time image segmentation of non-rigid objects in a
spatio-temporal domain with a stationary camera within an optical
high dynamic range environment. Camera, illumination and
segmentation techniques are discussed for image processing in
environments which are characterized by large intensity
fluctuations and hence a high optical dynamic range (HDR), in
particular for vehicle interior surveillance.

Since the introduction of the airbag in 1981 numberless lives were
saved and bad injuries were avoided. But in recent years the
airbag has frequently been in the headlines due to the increasing
number of injuries caused by it. %Children and infants are
%specially in danger if a passenger airbag deploys and hits the
%unprotected body.
To avoid these injuries a new generation of 'smart airba gs' has
been designed which shows the ability to inflate in multiple steps
and with different volumes. In order to determine the optimal
inflation mode for a crash it is necessary to consider information
about the interior situation and the occupants of the vehicle.
This thesis presents a real-time visual occupant detection and
classification system for advanced airbag deployment, utilizing a
custom CMOS camera and motion based image segmentation algorithms
for embedded systems under adverse illumination conditions.

A novel illumination method is presented which combines a set of
images flashed with different radiant intensities, which
significantly simplifies image segmentation in HDR environments.
With a constant exposure time for the imager a single image can be
produced with a compressed dynamic range and a simultaneously
reduced offset. This makes it possible to capture a vehicle
interior under adverse light conditions without using high dynamic
range cameras and without losing image detail. The expansion of
this active illumination experiment leads to a novel shadow
detection and removal technique that produces a shadow-free scene
by simulating an artificial infinite illuminant plane over the
field of view. Finally a shadowless image without loss of texture
details is obtained without any region extraction phase.


Furthermore, a texture based segmentation approach for stationary
cameras is presented which is neither effected by sudden
illumination changes nor by shadow effects. %This approach analysis
%the frame ratio instead of performing established background
%subtraction techniques.
