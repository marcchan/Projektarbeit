
\chapter{Einleitung}

\section{Motivation}

Die Bachelor- oder Masterarbeit ist ein besonders wichtiger Bestandteil des Studiums in den Abschlusssemestern. Sie stellt
eines der wenigen gegenständlich vorzeigbaren Arbeitsergebnisse des Studiums dar und ist auch deshalb, z.B.
bei Bewerbungen, von besonderer Bedeutung. Es liegt daher im Interesse eines jeden Studierenden, eine
sowohl inhaltlich als auch vom äußeren Erscheinungsbild her hohen Ansprüchen gerecht werdende
Dokumentation der Abschlussarbeit zu erstellen. Die nachfolgenden Hinweise sollen dabei Hilfestellung bieten.\\

Gegenstand des einleitenden Kapitels ist es, dem Leser einen Überblick über für die Arbeit relevante Grundlagen und verwandten Arbeiten zu geben. Folgende grundlegende Fragen sollten beantwortet werden:
\begin{itemize}
	\item Warum wird das Thema der Arbeit behandelt? 
	\item Wie lautet die genaue Aufgabendefinition?
	\item Welche Fragenstellungen werden in der Arbeit behandelt?	
	\item Wie ist die Arbeit strukturiert?  
\end{itemize}

Das Kapitel zur Einleitung dient dem Leser vorrangig als Entscheidungshilfe: Ist die Arbeit für mich überhaupt relevant? Werden für mich interessante Fragestellungen in der Arbeit behandelt? Sofern den Leser nur Teilergebnisse interessieren: Wo finde ich diese?

Die Motivation zur Projektarbeit kann vielfältig sein und sollte daher ausreichend begründet werden. Beispiele: Ein bestehendes Problem wurde durch den Stand der Technik bisher nicht oder nicht zufriedenstellend gelöst. Oder: Gegenstand der Arbeit ist eine besonders kostengünstige Lösung. Oder auch: Die Arbeit behandelt eine grundlegende Evaluation, um Möglichkeiten als auch Limitationen einer neuen Technologie aufzuzeigen.

Um dem Leser den logischen Aufbau und die Zusammenhänge einzelner Kapitel zu verdeutlichen ("roter Faden"), bieten sich beispielsweise Formulierungen an wie "Nachdem im zweiten Kapitel die Grundlagen der BlueTooth-Technologie behandelt wurde, wird im Kapitel 3 die Realisierung eines Systems zur Ortung von Satelliten mittels BlueTooth beschrieben. Anschließend folgt die Auswertung der in Kapitel 4 definierten Experimente zur Performanz des Systems".

\section{Formalien}

Dieser Abschnitt \dots

\section{Beispiel}

Dieser Abschnitt beinhaltet einen Ausschnitt aus der Ballade ''Erlkönig'' von Johann Wolfgang von Goethe~\cite{Goethe1782}.


{\it ''Wer reitet so spät durch Nacht und Wind? Es ist der Vater mit seinem Kind.
Er hat den Knaben wohl in dem Arm, Er faßt ihn sicher, er hält ihn warm.

Mein Sohn, was birgst du so bang dein Gesicht?
Siehst Vater, du den Erlkönig nicht!
Den Erlenkönig mit Kron' und Schweif?
Mein Sohn, es ist ein Nebelstreif.

Du liebes Kind, komm geh' mit mir!
Gar schöne Spiele, spiel ich mit dir,
Manch bunte Blumen sind an dem Strand,
Meine Mutter hat manch gülden Gewand.

Mein Vater, mein Vater, und hörest du nicht,
Was Erlenkönig mir leise verspricht?
Sei ruhig, bleibe ruhig, mein Kind,
In dürren Blättern säuselt der Wind.

Willst feiner Knabe du mit mir geh'n?
Meine Töchter sollen dich warten schön,
Meine Töchter führen den nächtlichen Reihn
Und wiegen und tanzen und singen dich ein.''
\dots
}

